\documentclass{article}
\usepackage{graphicx}
\usepackage[utf8]{inputenc}
\usepackage{mdframed}
\usepackage{minted}

\newenvironment{lcode}
  {\VerbatimEnvironment
   \begin{mdframed}
   \begin{minted}[breaklines, fontsize=\footnotesize]{python}}
  {\end{minted}
   \end{mdframed}}
\newenvironment{lresult}
  {\begin{mdframed}}
  {\end{mdframed}}

\graphicspath{ {./} }

\begin{document}

\section{Environment tags}

LitREPL for latex recognizes specifically named environments as code and result
sections. It doesn't really evaluate Tex commands so renaming those environments
wouldn't work. But we still need to introduce it to Latex so we start with a
newenvironment declarations.

\begin{verbatim}
\newenvironment{lcode}
  {\VerbatimEnvironment
   \begin{mdframed}
   \begin{minted}[breaklines, fontsize=\footnotesize]{python}}
  {\end{minted}
   \end{mdframed}}
\newenvironment{lresult}
  {\begin{mdframed}[leftline=false,rightline=false]}
  {\end{mdframed}}
\end{verbatim}

\section{Executing code snippets}

Executable code snippet is the content of the \texttt{lcode} environment.
Putting the cursor on it and typing the \texttt{:LitEval1} command runs it
it in a background Python interpreter.

\begin{lcode}
W='Hello, world!'
print(W)
\end{lcode}

\texttt{lresult} section next to the executable section is a result container.
LitREPL would paste here the output of above code snippet. The original content
of the section will be replaced.

\begin{lresult}
Hello, world!
\end{lresult}

Commented \texttt{lresult} environmet tags are still recognized as a section for
verbatim results. This way users can generate parts of the latex document.

\begin{lcode}
print('Hello, LitREPL.vim!')
\end{lcode}

\begin{mdframed}
%\begin{lresult}
Hello LitREPL.vim
%\end{lresult}
\end{mdframed}

\section{Invoking Mathplotlib code}

\begin{lcode}
import numpy as np
import matplotlib.pyplot as plt
X=np.array(range(10))
_=plt.plot(X, np.sin(X))
def texplot(f):
  plt.savefig(f)
  print("\\includegraphics{./"+f+"}")

texplot('_img.png')
\end{lcode}

\newcommand{\linplace}[2]{#2}
%\begin{lresult}
\includegraphics{\linplace{VAR}{_img.png}}
%\end{lresult}

\end{document}
